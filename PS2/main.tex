\documentclass[12pt, a4paper, hidelinks]{article}
\usepackage{myconfig}

\title{\textbf{Problem Set 2}}
\author{Lin Han}
\date{\today}

\begin{document}
\maketitle

\begin{problem}
Let the initial wavefunction be defined by
\begin{align*}
    \psi \(x, t = 0\) = \frac{1}{\sqrt{2}} \(\phi_1 \(x\) - \phi_3 \(x\)\)
\end{align*}
where $\phi_i \(x\)$ is the $i^{\text{th}}$ lowest-energy eigenstate of the particle in a box.
\begin{enumerate}[label=(\alph*)]
    \item Give the analytic expression for the probability density.
    \item Give the analytic expression for the expected evolution of the probability density.
    \item Explain how the expected evolution of the probability density relates to the expected 
    result for a nonstationary state.
    \item Give the analytic expression for the average position as a function of time.
\end{enumerate}
\end{problem}
\begin{solution}
    \begin{enumerate}[label=(\alph*)]
        \item 
        \begin{align*}
            \rho\(x\)&=\lno \Psi\(x,t=0\)\rno^2\\
            &=\frac{1}{2} \(\lno \phi_1\(x\)\rno^2+\lno \phi_3\(x\)\rno^2 
            -2\rm{Re}\(\phi_1^{\ast}\(x\)\phi_3\(x\)\)\)\\
            &=\frac{1}{l}\(\sin^2\(\frac{\pi x}{l}\)+\sin^2\(\frac{3\pi x}{l}\)-2
            \sin\(\frac{\pi x}{l}\)\sin\(\frac{3\pi x}{l}\)\)
        \end{align*}
        \item Given the initial wavefunction, we have:
        \begin{align*}
            \Phi\(x,t\)=\frac{1}{2}\(\phi_1\(x\)\e ^{-iE_1t/\hbar}-\phi_3\(x\)\e ^{-iE_3t/\hbar}\)
        \end{align*}
        then:
        \begin{align*}
            \rho\(x,t\)&=\lno \Psi\(x,t\)\rno^2\\
            &=\frac{1}{l}\(\sin^2\(\frac{\pi x}{l}\)+\sin^2\(\frac{3\pi x}{l}\)-\rm{Re}\(2
            \sin\(\frac{\pi x}{l}\)\sin\(\frac{3\pi x}{l}\)\e^{\frac{i}{\hbar}\(E_1-E_3\)t}\)\)\\
            &=\frac{1}{l}\(\sin^2\(\frac{\pi x}{l}\)+\sin^2\(\frac{3\pi x}{l}\)-2
            \sin\(\frac{\pi x}{l}\)\sin\(\frac{3\pi x}{l}\)\cos\(\frac{\(E_1-E_3\)}{\hbar}t\)\)
        \end{align*}
        where the evolution is:
        \begin{align*}
            \frac{\d \rho\(x,t\)}{\d t}=\frac{2}{l}\cdot\frac{E_1-E_3}{\hbar}
            \sin\(\frac{\pi x}{l}\)\sin\(\frac{3\pi x}{l}\)\sin\(\frac{\(E_1-E_3\)}{\hbar}t\)
        \end{align*}
        \item The system is in its nonstationary state, which means it can be expressed as superposition 
        of stationary states. The expected evolution of the probability density is related to the 
        interference of the stationary states, and is not 0.
        \item 
        \begin{align*}
            \la x\ra&=\int_{0}^{l}\d x\ x\rho\(x,t\)\\
            &=\frac{l}{2}
        \end{align*}
        which means superposition does not affect the average position.
    \end{enumerate}
\end{solution}
\begin{problem}
    Let the initial wavefunction be defined as the Gaussian wavepacket
    \begin{align*}
        \Psi\(x,t=0\)=N\e^{-a^2\(x-L/2\)^2/2}
    \end{align*}
    where $N$ is the normalization coefficient, $L=5$\AA, and $a=4$\AA$^{-1}$.
    \begin{enumerate}[label=(\alph*)]
        \item Determine analytically whether $\Psi\(x,t=0\)$ is a stationary state for a particle-in-a-box 
        potential.
        \item Using the accompanying Jupyter notebook:
        \begin{enumerate}[label=\roman*.]
            \item Explain what the function $f\(N\)=\sum_{i=1}^{N}c_i^{\ast}c_i$ represents.
            \item Plot $f\(N\)$ for the given expansion coefficients $c_i$.
            \item Calculate $\lim_{N\to \infty}f\(N\)$.
            \item Explain what $f\(N\)$ says about the number of expansion terms required to accurately 
            simulate the wavefunction..
            \item Plot $\lno\Psi\(x,t\)\rno^2$ at several choices of $t$ to show how the wavefunction changes in time.
            \item Explain what $\lno\Psi\(x,t\)\rno^2$ tells us about the motion os a particle in a box.
            \item Plot $\la x\(t\)\ra$ and $\la\(x-L/2\)^2\(t\)\ra$.
            \item Explain the significance of $\la x\(t\)\ra$ and $\la\(x-L/2\)^2\(t\)\ra$ and their 
            relationships to $\lno\Psi\(x,t\)\rno^2$.
        \end{enumerate}
    \end{enumerate}
\end{problem}
\begin{solution}
    \begin{enumerate}[label=(\alph*)]
        \item 
        From normalization, we have:
        \begin{align*}
            N=\(\frac{a^2}{\pi}\)^{1/4}
        \end{align*}
        For an in-box particle at $t=0$, the wavefunction could be expressed as:
        \begin{align*}
            \Psi\(x,t=0\)=\sum_{i}a_i\phi_i\(x\)
        \end{align*}
        To determine whether $\Psi\(x,t=0\)$ is a stationary state, we need to check whether it is a superposition 
        of eigenstates. Solve for $c_i$, we get:
        \begin{align*}
            c_i&=\int_{0}^{l}\d x \ \Psi\(x,t=0\)\cdot\sqrt{\frac{2}{l}}\sin\(\frac{i\pi x}{l}\)\\
            &=\int_{-\infty}^{\infty}\d x \ \Psi\(x,t=0\)\cdot\sqrt{\frac{2}{l}}\sin\(\frac{i\pi x}{l}\)\\
            &=2\frac{\pi^{1/4}}{\sqrt{la}}\sin\(\frac{i\pi}{2}\)\e ^{-\frac{i^2\pi^2}{2a^2 l^2}}
        \end{align*}
        when i is odd, $a_i$ is not zero, which means $\Psi\(x,t=0\)$ is not a stationary state.
        \item 
        \begin{enumerate}[label=\roman*.]
            \item If we expand the wavefunction with particle-in-box eigenstates, it can be written as:
            \begin{align*}
                \Psi\(x,t=0\)=\sum_{i}c_i\phi_i\(x\)
            \end{align*}
            For normalization, we have:
            \begin{align*}
                \sum_{i=1}^{\infty}\lno c_i\rno^2=1
            \end{align*}
            which means $\lno c_i\rno^2$ is the probability of the particle in the $i^{\text{th}}$ eigenstate.
            So $f\(N\)=\sum_{i}^{N}\lno c_i\rno^2$ is the probability of the particle in the first eigenstate to the $N^{\text{th}}$
            eigenstate.
            \item We can plot $f\(N\)$ as:
            \begin{figure}[H]
                \centering
                \includegraphics[width=0.8\textwidth]{fig1}
                \caption{Plot of $f\(N\)$}
                \label{fig1}
            \end{figure}
            \item When $N\to \infty$, we can use intergral to calculate the limit:
            \begin{align*}
                \lim_{N\to \infty}f\(N\)&=\int_{0}^{\infty}\d i\ \lno c_i\rno^2\\
                &=\int_{0}^{\infty}\d i\ 4\(\frac{\pi}{l^2 a^2}\)^{1/2}\sin^2\(\frac{i\pi}{2}\)\e ^{-\frac{i^2\pi^2}{a^2l^2}}\\
                &=1-\e^{-\frac{a^2l^2}{4}}
            \end{align*}
            assume that $al\gg 1$, we have:
            \begin{align*}
                \lim_{N\to \infty}f\(N\)\approx 1
            \end{align*}
            \item From the plot in Fig.\ref{fig1}, we can see that when the number of extension terms is small, 
            especially when $N< 20$, $f\(N\)< 0.8$, which means there's only 80\% probability of the particle
            in the first 20 eigenstates. So to accurately simulate the wavefunction, we need to include more expansion terms. 
            One great idea tis to connect the suitable number of expansion terms with $la$, which means to better fit 
            the wavefunction, we should make $\frac{N}{al}\gg 1$.
            \item Plots are as follows:
            \begin{figure}[H]
                \centering
                \includegraphics[width=0.8\textwidth]{fig2}
                \caption{Plot of $\lno\Psi\(x,t\)\rno^2$}
                \label{fig2}
            \end{figure}
            \item Because the wavefunction is not a stationary state, the probability density will change in time. Due to the superposition,
            the density will form a peridic structure.
            \item Plots are as follows:
            \begin{figure}[H]
                \centering
                \includegraphics[width=0.8\textwidth]{fig3}
                \caption{Plot of $\la x\(t\)\ra$}
                \label{fig3}
            \end{figure}
            \begin{figure}[H]
                \centering
                \includegraphics[width=0.8\textwidth]{fig4}
                \caption{Plot of $\la\(x-L/2\)^2\(t\)\ra$}
                \label{fig4}
            \end{figure}
            \item From these plots, we can see that $\la x\(t\)\ra$ is constant with $t$, which means 
            all of the eigenstates are centered at the same position, i.e., the middle of the box. And 
            superposition does not affect the average position. $\la\(x-L/2\)^2\(t\)\ra$ arises from 0 
            to a constant value, which means that superposition affects the density of the particle, but 
            with enough time, the density will become uniform statistically, which makes $\la\(x-L/2\)^2\(t\)\ra$ 
            a constant and same value as when the density is uniform.
        \end{enumerate}
\end{enumerate}
\end{solution}
\end{document}
